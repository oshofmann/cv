%%%%%%%%%%%%%%%%%%%%%%%%%%%%%%%%%%%%%%%%%%%%%%%%%%%%%%%%%%%%%%%%%%%%%%%%
%%%%%%%%%%%%%%%%%%%%%% Simple LaTeX CV Template %%%%%%%%%%%%%%%%%%%%%%%%
%%%%%%%%%%%%%%%%%%%%%%%%%%%%%%%%%%%%%%%%%%%%%%%%%%%%%%%%%%%%%%%%%%%%%%%%

%%%%%%%%%%%%%%%%%%%%%%%%%%%%%%%%%%%%%%%%%%%%%%%%%%%%%%%%%%%%%%%%%%%%%%%%
%% NOTE: If you find that it says                                     %%
%%                                                                    %%
%%                           1 of ??                                  %%
%%                                                                    %%
%% at the bottom of your first page, this means that the AUX file     %%
%% was not available when you ran LaTeX on this source. Simply RERUN  %%
%% LaTeX to get the ``??'' replaced with the number of the last page  %%
%% of the document. The AUX file will be generated on the first run   %%
%% of LaTeX and used on the second run to fill in all of the          %%
%% references.                                                        %%
%%%%%%%%%%%%%%%%%%%%%%%%%%%%%%%%%%%%%%%%%%%%%%%%%%%%%%%%%%%%%%%%%%%%%%%%

%%%%%%%%%%%%%%%%%%%%%%%%%%%% Document Setup %%%%%%%%%%%%%%%%%%%%%%%%%%%%

% Don't like 10pt? Try 11pt or 12pt
\documentclass[10pt]{article}

% This is a helpful package that puts math inside length specifications
\usepackage{calc}

% Simpler bibsection for CV sections
% (thanks to natbib for inspiration)
\makeatletter
\newlength{\bibhang}
\setlength{\bibhang}{1em}
\newlength{\bibsep}
 {\@listi \global\bibsep\itemsep \global\advance\bibsep by\parsep}
\newenvironment{bibsection}
    {\list{}{%
        \setlength{\leftmargin}{\bibhang}%
        \setlength{\itemindent}{-\leftmargin}%
        \setlength{\itemsep}{\bibsep}%
        \setlength{\parsep}{\z@}%
        }}
    {\endlist}
\newenvironment{bibsectionfirst}
    {\minipage[t]{\linewidth}\list{}{%
        \setlength{\leftmargin}{\bibhang}%
        \setlength{\itemindent}{-\leftmargin}%
        \setlength{\itemsep}{\bibsep}%
        \setlength{\parsep}{\z@}%
        }}
    {\endlist\endminipage}
\makeatother

% Layout: Puts the section titles on left side of page
\reversemarginpar

%
%         PAPER SIZE, PAGE NUMBER, AND DOCUMENT LAYOUT NOTES:
%
% The next \usepackage line changes the layout for CV style section
% headings as marginal notes. It also sets up the paper size as either
% letter or A4. By default, letter was used. If A4 paper is desired,
% comment out the letterpaper lines and uncomment the a4paper lines.
%
% As you can see, the margin widths and section title widths can be
% easily adjusted.
%
% ALSO: Notice that the includefoot option can be commented OUT in order
% to put the PAGE NUMBER *IN* the bottom margin. This will make the
% effective text area larger.
%
% IF YOU WISH TO REMOVE THE ``of LASTPAGE'' next to each page number,
% see the note about the +LP and -LP lines below. Comment out the +LP
% and uncomment the -LP.
%
% IF YOU WISH TO REMOVE PAGE NUMBERS, be sure that the includefoot line
% is uncommented and ALSO uncomment the \pagestyle{empty} a few lines
% below.
%

%% Use these lines for letter-sized paper
\usepackage[paper=letterpaper,
            %includefoot, % Uncomment to put page number above margin
            marginparwidth=1.2in,     % Length of section titles
            marginparsep=.05in,       % Space between titles and text
            margin=1in,               % 1 inch margins
            includemp]{geometry}

%% Use these lines for A4-sized paper
%\usepackage[paper=a4paper,
%            %includefoot, % Uncomment to put page number above margin
%            marginparwidth=30.5mm,    % Length of section titles
%            marginparsep=1.5mm,       % Space between titles and text
%            margin=25mm,              % 25mm margins
%            includemp]{geometry}

%% More layout: Get rid of indenting throughout entire document
\setlength{\parindent}{0in}

%% This gives us fun enumeration environments. compactitem will be nice.
\usepackage{paralist}

%% Reference the last page in the page number
%
% NOTE: comment the +LP line and uncomment the -LP line to have page
%       numbers without the ``of ##'' last page reference)
%
% NOTE: uncomment the \pagestyle{empty} line to get rid of all page
%       numbers (make sure includefoot is commented out above)
%
\usepackage{fancyhdr,lastpage}
\pagestyle{fancy}
%\pagestyle{empty}      % Uncomment this to get rid of page numbers
\fancyhf{}\renewcommand{\headrulewidth}{0pt}
\fancyfootoffset{\marginparsep+\marginparwidth}
\newlength{\footpageshift}
\setlength{\footpageshift}
          {0.5\textwidth+0.5\marginparsep+0.5\marginparwidth-2in}
\lfoot{\hspace{\footpageshift}%
       \parbox{4in}{\, \hfill %
                    \arabic{page} of \protect\pageref*{LastPage} % +LP
%                    \arabic{page}                               % -LP
                    \hfill \,}}

% Finally, give us PDF bookmarks
\usepackage{color,hyperref}
\definecolor{darkblue}{rgb}{0.0,0.0,0.0}
\hypersetup{colorlinks,breaklinks,
            linkcolor=darkblue,urlcolor=darkblue,
            anchorcolor=darkblue,citecolor=darkblue}

%%%%%%%%%%%%%%%%%%%%%%%% End Document Setup %%%%%%%%%%%%%%%%%%%%%%%%%%%%


%%%%%%%%%%%%%%%%%%%%%%%%%%% Helper Commands %%%%%%%%%%%%%%%%%%%%%%%%%%%%

% The title (name) with a horizontal rule under it
%
% Usage: \makeheading{name}
%
% Place at top of document. It should be the first thing.
\newcommand{\makeheading}[1]%
        {\hspace*{-\marginparsep minus \marginparwidth}%
         \begin{minipage}[t]{\textwidth+\marginparwidth+\marginparsep}%
                {\large \bfseries #1}\\[-0.15\baselineskip]%
                 \rule{\columnwidth}{1pt}%
         \end{minipage}}

% The section headings
%
% Usage: \section{section name}
%
% Follow this section IMMEDIATELY with the first line of the section
% text. Do not put whitespace in between. That is, do this:
%
%       \section{My Information}
%       Here is my information.
%
% and NOT this:
%
%       \section{My Information}
%
%       Here is my information.
%
% Otherwise the top of the section header will not line up with the top
% of the section. Of course, using a single comment character (%) on
% empty lines allows for the function of the first example with the
% readability of the second example.
\renewcommand{\section}[2]%
        {\pagebreak[2]\vspace{1.3\baselineskip}%
         \phantomsection\addcontentsline{toc}{section}{#1}%
         \hspace{0in}%
         \marginpar{
         \raggedright \scshape #1}#2}

% An itemize-style list with lots of space between items
\newenvironment{outerlist}[1][\enskip\textbullet]%
        {\begin{itemize}[#1]}{\end{itemize}%
         \vspace{-.6\baselineskip}}

% An environment IDENTICAL to outerlist that has better pre-list spacing
% when used as the first thing in a \section
\newenvironment{lonelist}[1][\enskip\textbullet]%
        {\vspace{-\baselineskip}\begin{list}{#1}{%
        \setlength{\partopsep}{0pt}%
        \setlength{\topsep}{0pt}}}
        {\end{list}\vspace{-.6\baselineskip}}

% An itemize-style list with little space between items
\newenvironment{innerlist}[1][\enskip\textbullet]%
        {\begin{compactitem}[#1]}{\end{compactitem}}

% To add some paragraph space between lines.
% This also tells LaTeX to preferably break a page on one of these gaps
% if there is a needed pagebreak nearby.
\newcommand{\blankline}{\quad\pagebreak[2]}

% 

%%%%%%%%%%%%%%%%%%%%%%%% End Helper Commands %%%%%%%%%%%%%%%%%%%%%%%%%%%

%%%%%%%%%%%%%%%%%%%%%%%%% Begin CV Document %%%%%%%%%%%%%%%%%%%%%%%%%%%%

\usepackage{comment}

\begin{document}
\makeheading{Owen S.~Hofmann}

\section{Contact Information}
%
% NOTE: Mind where the & separators and \\ breaks are in the following
%       table.
%
% ALSO: \rcollength is the width of the right column of the table
%       (adjust it to your liking; default is 1.85in).
%
\newlength{\rcollength}\setlength{\rcollength}{2in}%
%
\begin{tabular}[t]{@{}p{\textwidth-\rcollength}p{\rcollength}}
\href{http://www.cs.utexas.edu/}{Department of Computer Science} &
   \textit{Phone:} (512) 576-2109 \\
\href{http://www.utexas.edu/}{The University of Texas at Austin} &
   \textit{E-mail:} \href{mailto:osh@cs.utexas.edu}{osh@cs.utexas.edu} \\
1616 Guadalupe, Suite 2.408 & \textit{Web:}
   \href{http://www.cs.utexas.edu/~osh/}{www.cs.utexas.edu/{\footnotesize$\sim$}osh/} \\
Austin, TX 78701 & \\
\end{tabular}

\section{Research Interests}
%
Operating systems, architecture, parallel programming

\section{Education}
%
\href{http://www.utexas.edu/}{\textbf{The University of Texas at Austin}},
Austin, Texas
\begin{outerlist}

\item[] Ph.D., Computer Science, expected 2013
        \begin{innerlist}
        \item Adviser:
              \href{http://www.cs.utexas.edu/~witchel/}
                   {Emmett Witchel}
        \item Thesis: Rethinking Operating System Trust
        \end{innerlist}

\end{outerlist}

\bigskip

\href{http://www.amherst.edu}{\textbf{Amherst College}}, Amherst,
Massachusetts
\begin{outerlist}
\item[] B.A. \emph{Magna cum Laude}, Computer Science, May 2006
        \begin{innerlist}
        \item Thesis: \emph{Reference Trace Reduction via Recency
Distribution Sampling}
        \end{innerlist}

\end{outerlist}

\section{Publications}
\begin{bibsectionfirst}

\item Owen S. Hofmann, Sangman Kim, Alan M. Dunn, Michael Z. Lee, Emmett
Witchel.
Inktag: Secure Applications on an Untrusted Operating System.
To appear in \emph {Proceedings of the 18th International Conference on
Architectural Support for Programming Languages and Operating Systems
(ASPLOS), Houston, TX, March 2013}

\end{bibsectionfirst}
\begin{bibsection}

\item Edmund B. Nightingale, Jeremy Elson, Jinliang Fan, Owen Hofmann, Jon
Howell, and Yutaka Suzue.
Flat Datacenter Storage. 
In \emph{Proceedings of the 10th USENIX Symposium on Operating Systems
Design and Implementation (OSDI), Hollywood, CA, October 2012}

\item Owen S. Hofmann, Alan Dunn, Sangman Kim, Indrajit Roy, Emmett Witchel.
Ensuring Operating System Kernel Integrity with OSck.
In \emph{Proceedings of the 16th International Conference on Architectural
Support for Programming Languages and Operating Systems (ASPLOS), Newport
Beach, CA, March 2011} 

\item Sangman Kim, Michael Z. Lee, Alan M. Dunn, Owen S. Hofmann, Xuan
Wang, Emmett Witchel, Donald E. Porter.
Improving Server Applications with System Transactions.
In \emph{Proceedings of the 7th ACM European Conference on Computer Systems
(EuroSys), Bern, Switzerland, April 2012}

\item Alan M. Dunn, Owen S. Hofmann, Brent Waters, Emmett Witchel.
Cloaking Malware with the Trusted Platform Module.
In \emph{Proceedings of the 20th USENIX Security Symposium, San Francisco,
CA, August 2011}

\item Scott Wolchok, Owen S. Hofmann,  Nadia Heninger,  Edward W. Felten,
J. Alex Halderman, Christopher J. Rossbach, Brent Waters, Emmett Witchel.
Defeating Vanish with Low-Cost Sybil Attacks Against Large DHTs.
 In \emph{Proceedings of the 17th Network and Distributed System Security
Symposium (NDSS), San Diego, California, February, 2010}


\item Christopher J. Rossbach, Owen S. Hofmann, Emmett Witchel.
Is Transactional Memory Programming Actually Easier?
In \emph{Proceedings of the  15th ACM SIGPLAN Annual Symposium on
Principles and Practice of Parallel Programming (PPoPP), Bangalore, India,
January 2010} 
 
\item Donald E. Porter, Owen S. Hofmann, Christopher J. Rossbach, Alexander
Benn, Emmett Witchel.
Operating System Transactions.
In \emph{Proceedings of the 22nd ACM Symposium on Operating Systems
Principles (SOSP), Big Sky, MT, October 2009} 
 
\item Christopher J. Rossbach, Owen S. Hofmann, Emmett Witchel.
Is Transactional Memory Programming Actually Easier?
In \emph{Proceedings of the  8th Annual Workshop on Duplicating,
Deconstructing, and Debunking (WDDD), Austin, Texas June 2009} 
 
\item Owen S. Hofmann, Christopher J. Rossbach, Emmett Witchel.
Maximum Benefit from a Minimal HTM.
In \emph{Proceedings of the 14th International Conference on Architectural
Support for Programming Languages and Operating Systems (ASPLOS),
Washington DC, March 2009} 
 
\item Christopher J. Rossbach, Owen S. Hofmann, Donald E. Porter, Hany E.
Ramadan, Aditya Bhandari, Emmett Witchel.
TxLinux/MetaTM: Transactional Memory and the Operating System.
In \emph{Communications of the ACM, September 2008} 
 
\item Hany E. Ramadan, Christopher J. Rossbach, Donald E. Porter, Owen S.
Hofmann, Aditya Bhandari, Emmett Witchel.
MetaTM/TxLinux: Transactional Memory For An Operating System.
 In \emph{IEEE Micro, Jan/Feb 2008}
 
\item Christopher J. Rossbach, Owen S. Hofmann, Donald E. Porter, Hany E.
Ramadan, Aditya Bhandari, Emmett Witchel.
TxLinux: Using and Managing Hardware Transactional Memory in an
Operating System.
In \emph{Proceedings of the 21st ACM Symposium on Operating Systems
Principles (SOSP), Stevenson, WA October 2007} 
 
\item Owen S. Hofmann, Donald E. Porter, Hany E. Ramadan, Christopher J.
Rossbach, Emmett Witchel.
Solving Difficult HTM Problems Without Difficult Hardware.
In \emph{Proceedings of the 2nd Workshop on Transactional Computing
(TRANSACT), Portland, OR August 2007.} 
 
\item Hany E. Ramadan, Christopher J. Rossbach, Donald E. Porter, Owen S.
Hofmann, Aditya Bhandari, Emmett Witchel.
MetaTM/TxLinux: Transactional Memory For An Operating System.
In \emph{Proceedings of the International Symposium on Computer
Architecture 2007 (ISCA), San Diego, CA June 2007.} 
 
\item Donald E. Porter, Owen S. Hofmann, Emmett Witchel.
Is the Optimism in Optimistic Concurrency Warranted? 
In \emph{Proceedings of the 11th Workshop on Hot Topics in Operating
Systems (HotOS) San Diego, CA May 2007.} 
 
\item Sasha Ames, Nikhil Bobb, Kevin Greenan, Owen Hofmann, Mark W. Storer,
Carlos Maltzahn, Ethan L. Miller, Scott A. Brandt.
LiFS: An Attribute-Rich File System for Storage Class Memories.
In \emph{Proceedings of the 23rd IEEE / 14th NASA Goddard Conference on
Mass Storage Systems and Technologies (MSST), College Park, MD May
2006.} 
\end{bibsection}



\section{Awards}
%
CACM Invited Article for September 2008: \\
Christopher J. Rossbach,
Owen S. Hofmann, Donald E. Porter, Hany E. Ramadan, Aditya Bhandari,
Emmett Witchel. TxLinux: Using and Managing Transactional Memory in
an Operating System

\bigskip

IEEE Top Pick for 2008: \\
Hany E. Ramadan, Christopher J. Rossbach, Donald E. Porter,
Owen S. Hofmann, Aditya Bhandari, Emmett Witchel.
MetaTM/TxLinux: Transactional Memory for an Operating System

\bigskip

SOSP 2007 Audience Choice Best Paper: \\
Christopher J. Rossbach,
Owen S. Hofmann, Donald E. Porter, Hany E. Ramadan, Aditya Bhandari,
Emmett Witchel. 
TxLinux: Managing and Supporting Hardware Transactional Memory in an
Operating
System

\bigskip

Microelectronics and Computer Development (MCD) Graduate Fellowship,
The University of Texas at Austin, August 2006 to May 2007

\pagebreak

\section{Academic Experience}
\href{http://www.cs.utexas.edu}{\textbf{The University of Texas at Austin,
Department of Computer Science}}
\begin{outerlist}

\item[] \textit{Research Assistant}%
    \hfill \textbf{May 2007 to present}
    \begin{innerlist}
        \item TxLinux 2.6 
        \begin{innerlist}
           \item Worked on team to develop TxLinux, the first operating system
to use hardware transactional memory for synchronization
           \item Designed and implemented \emph{cooperative transactional
spinlocks}, forming the basis for large-scale automatic conversion of the
Linux kernel to use hardware transactions
           \item Published in ISCA '07 and SOSP '07
        \end{innerlist}
        \item TxLinux 2.4
        \begin{innerlist}
           \item Applied techniques from TxLinux 2.6 to Linux 2.4 to
demonstrate the benefit of hardware transactional memory for
improving coarse-grained synchronization performance in the operating
system
           \item Introduced \emph{transaction ordering}, a novel technique for
unifying hardware and software transactional memory in user applications
           \item Published in ASPLOS '09
        \end{innerlist}
        \item TxOS
        \begin{innerlist}
           \item Worked on team to develop TxOS, a version of Linux
providing \emph{system transactions} to ensure atomic, isolated and
consistent updates to diverse system resources
           \item Modified the ext3 file system to provide durable
transactional updates
           \item Published in SOSP '09
        \end{innerlist}
        \item Unvanish
        \begin{innerlist}
           \item Identified vulnerabilities in the Vanish system for
self-destructing data
           \item Designed and implemented Unvanish, a system to recover
data protected by Vanish
           \item Published in NDSS '10
         \end{innerlist}
         \item OSck
         \begin{innerlist}
            \item Designed and implemented a system for efficient
hypervisor-based kernel rootkit detection based on the Linux KVM hypervisor.
            \item Published in ASPLOS '11
         \end{innerlist}
         \item InkTag
         \begin{innerlist}
            \item Designed and implemented InkTag, a system for securely
executing applications under an untrusted operating system.
            \item To appear in ASPLOS '13
         \end{innerlist}
    \end{innerlist}
\item[] \textit{Assistant Instructor}%
        \hfill \textbf{January 2012 to May 2012}
        \begin{innerlist}
           \item CS429: Computer Organization and Architecture
           \begin{innerlist}
              \item hey
           \end{innerlist}
        \end{innerlist}
\item[] \textit{Teaching Assistant}%
        \hfill \textbf{August 2007 to December 2007}
        \begin{innerlist}
           \item CS372H: Operating Systems Honors
           \begin{innerlist}
              \item Held office hours, graded undergraduate lab assignments and exams
              \item Developed new lab assignments on synchronization
              \item Designed a user study based on student experience and
performance with synchronization labs
              \item Study results later published in PPoPP '10
           \end{innerlist}
        \end{innerlist}
\end{outerlist}

\bigskip

\href{http://research.microsoft.com}{\textbf{Microsoft Research}}
\begin{outerlist}

\item[] \textit{Intern}
    \hfill \textbf{June 2010 to August 2010}
    \begin{innerlist}
        \item Flat Datacenter Storage
        \begin{innerlist}
           \item Worked on team developing initial implementation of Flat
Datacenter Storage (FDS)
           \item Built initial implementation of failure recovery within
FDS
           \item FDS currently holds records in both ``Indy'' and
``Daytona'' categories for the MinuteSort benchmark
(\href{http://www.sortbenchmark.org}{www.sortbenchmark.org}).
           \item Published in OSDI '12
        \end{innerlist}
     \end{innerlist}
\end{outerlist}

\begin{comment}

\section{Technical\\Skills}
C, C++, x86 assembly, Java, Python, PHP

\bigskip

Kernel hacking, parallel programming

\bigskip

Data analysis and graphing with the R
software environment for statistical computing

\section{References}
\end{comment}

\end{document}

%%%%%%%%%%%%%%%%%%%%%%%%%% End CV Document %%%%%%%%%%%%%%%%%%%%%%%%%%%%%
