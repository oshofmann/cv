\documentclass{article}

\usepackage{xspace}

\newcommand{\osck}{OSck\xspace}
\newcommand{\inktag}{InkTag\xspace}
\newcommand{\codeid}[1]{{\fontsize{9pt}{12pt}\tt #1}}

\begin{document}
The operating system is the primary mediator between computer hardware and
other layers of the system software stack, and operating systems research
has often focused on supporting and integrating new and emerging trends in
computer architecture. With my own interest in both operating systems as
well as low-level implementation, my research has continued this tradition.
However, key to my research has been using emerging architectural trends
not only as a new target to be supported, but also as a starting point for
high-level thought about core OS principles, such as the trust relationship
between applications and the operating system. In addition, I have
concentrated on projects with serious implementations on production OS
kernels.

In the past, my work has focused on hardware transactional memory (HTM),
which aims to simplify concurrent programming by 

My work with hardware transactional memory (HTM) and TxLinux~\cite{txlinux}
used the complexity of the kernel as a proving ground for solving difficult
transactional memory problems, such as invoking I/O operations within a
transaction. The ability to run the same workload across different version
of the kernel also provides a unique opportunity to evaluate the core
claims of transactional memory: the ability to write simpler code with
increased synchronization performance.

While HTM has yet to see widespread adoption, architectural support for
virtualization is now a key selling point for modern processors.

{\bigskip \noindent\bf Past: Hardware transactional memory and the Linux
kernel}

\noindent Concurrent programming is difficult, but necessary. Hardware
Transactional Memory~\cite{herlihy} (HTM) proposes to simplify concurrent
programming

My research focuses on hardware and software assistance for writing
concurrent programs. Recent hardware trends have had manufacturers scaling
the number of processing cores on a single chip, rather than the speed of a
single core.  This trend has shifted the burden of application performance
from the hardware designer to the programmer, who must now write parallel
code to take advantage of multiple cores. However, there has been
relatively little development in programming parallel architectures. In
particular, mutual exclusion through locking remains the primary form of
synchro- nization for parallel programs. Locking introduces a number of
well-known correctness and performance issues in parallel code.
High-performance parallel code requires fine-grained locking, and advanced
reasoning about locking discipline to avoid deadlock.

The next generation of parallel hardware will require simpler methods for
writing high-performance parallel programs.  Hardware transactional memory
(HTM) is a proposal that has attracted attention as a powerful
synchronization primitive that is easier to reason about than locks and can
be implemented with moderate hardware support. Transactional memory (TM) is
a form of optimistic concurrency. To synchronize a program, the programmer
need only delimit the critical regions of code that access shared data.
These regions would otherwise be guarded by locks or by another form of
synchronization.  Critical regions of different threads speculatively
execute in parallel. TM enforces a serializable interleaving of critical
regions by buffering writes and detecting conflicting data accesses.
Conflicting critical regions roll back and are re- executed. Because
critical regions will execute in parallel if it is safe, the programmer
does not need to manually add fine-grained synchronization to get good
performance. Hardware transactional memory is implemented by adding a small
amount of state to processor caches and cache coherence mechanisms, which
already manage data sharing between processors.

As with any new architectural feature, HTM has a high barrier to adoption
in production processors. Processor manufacturers are understandably
reluctant to spend both development effort and die space on features that
may not enjoy widespread use. Recent directions in HTM research have
exacerbated this problem, proposing solutions that manipulate bitfields
several kilobits in size with every memory operation, or that require
modifications to the entire memory hierarchy.

My research focuses on the benefit provided by a simple HTM design that
does not support transactions that overflow hardware resources or perform
complicated operations such as I/O. These exceptional situations are the
subject of signif- icant HTM research. However, these problems can be
addressed via minimal modifications to the standard transactional ISA,
without requiring baroque hardware design.

TxLinux 2.6 (SOSP ’07) is a transactional version of the Linux 2.6 kernel
that introduces Cooperative Transactional Spinlocks (cxspinlocks).
Cxspinlocks are a synchronization primitive that can use either locking or
transactions to protect a single critical region, and provides fairness
between both. A transaction that overflows hardware resources or performs
I/O restarts and falls back on locking, and threads attempting to use
either locks or transactions for synchronization contend fairly for a
single critical region. Implementing the ISA support for both correct and
fair cooperation between transactions and locks does not require changing
the hardware structures that support HTM.

TxLinux 2.4 (ASPLOS ’09) shows that a simple HTM design can have
significant performance benefit for a large, complex codebase. Linux 2.4
uses a coarse-grained locking structure that is much simpler than the
highly engineered synchronization of Linux 2.6, and as a result exhibits
significantly less scalability. TxLinux 2.4 uses transactions to improve
performance over Linux 2.4 by as much as 40%. Even a simple HTM design can
significantly soften the tradeoff between synchronization performance and
complexity. In addition, the work extends the cxspinlock technique to
blocking primitives such as mutexes, and introduces a new technique for
transitioning between transactional and locking critical regions without
requiring expensive transaction restarts. Both of these developments use
software to increase the utility of hardware transactions without requiring
a new hardware design.

Cxspinlocks provide a compelling programming model for the Linux kernel,
greatly improving the performance of simple, coarse-grained
synchronization. To provide the complete benefits of transactions (atomic
critical sections without explicit locking) to application developers, the
work in ASPLOS ’09 proposes transaction ordering, a novel technique for
combining hardware and software synchronization. Transaction ordering
allows a wide variety of HTM designs to syn- chronize access to the same
data as a wide variety of software synchronization, such as locking or
software transactional 1 memory (STM). Software can then handle exceptional
transactions that cannot be executed correctly in hardware, such as those
that overflow hardware resources. Existing research has proposed hybrid
hardware-software designs, but all require a close pairing of hardware and
software systems.


{\bigskip \noindent \bf Current: Rethinking operating system trust}

\noindent The operating system has long been the default root of trust for
all applications running on a system. However, this trust relationship is
less than ideal. The operating system is shared between all applications,
significantly expanding the threat of individual application
vulnerabilities. If an application is compromised, and can compromise the
OS, or elevate its privilege, what had been a vulnerability in a single
application becomes a vulnerability for all applications, because the OS
can completely control application behavior.

%Although securing operating systems is critical, operating systems are
%difficult pieces of software to secure.  Modern operating systems consist
%of millions of lines of code, written by numerous parties. A security bug
%in a single line of code often compromises the system in its entirety.
%Furthermore, the operating system's attack surface as seen from an
%application is the system call interface. Over time, the system call
%interface has grown from a handful of simple calls for operating on files
%(such as \codeid{open}, \codeid{read}, and \codeid{write}), to a vast
%library of functionality for operating on infrequently used OS services.
%Drivers often have the ability to arbitrarily extend this interface even
%further, through calls such as \codeid{ioctl}.

As OS security has become more perilous, hypervisors have reemerged as a
powerful and useful layer in the system software stack. The utility of
software-based virtualization for x86 processors has led to increases in
performance via paravirtualization, and finally to full hardware
virtualization support in current processor generations. In a virtualized
system, the hypervisor becomes the new software root of trust, and
hypervisors are significantly better candidates for this role than
operating systems. Hypervisors contain fewer lines of code than a typical
operating system. Also, the hypervisor interface is a hardware interface,
which is far simpler and easier to make secure than the hundreds of system
calls exported by the operating system.

%From 2010 to 2012, a
%search of the National Vulnerability Database returns 3 exploits for Xen and
%13 exploits for KVM that have an impact worse than denial of service. By
%contrast, there were 26 such vulnerabilities published for the Linux kernel
%in May and June of 2012 alone, of which only 4 were for device drivers.

From the perspective of the application, however, hypervisors only increase
the size of the trusted computing base. Applications still rely on and are
entirely controlled by the same operating system as before. This thesis
investigates ways in which a trusted hypervisor can improve this situation.
\textbf{\osck }is a system which attempts to increase application trust in
the OS, by detecting \emph{kernel rootkits}, malicious pieces of software
that can be impossible to detect from an application, or even the kernel,
because they modify kernel functionality. \textbf{\inktag }removes the
necessity for an application to trust the OS at all, by isolating an
application from an untrusted operating system, and giving it the tools to
validate that the operating system is behaving faithfully (for example by
mapping the correct files into the right part of the application's address
space).

Hypervisor support for rootkit detection, and hypervisor support for
untrusted operating systems have been proposed in previous work. The key
insight introduced with this thesis is the significant power afforded to
such systems by leveraging information from the potentially malicious
operating system. For example, most kernel rootkits modify OS functionality
by replacing function pointers in kernel objects. Validating that the
kernel will invoke the right function involves tracing all possible paths
through tens to hundreds of thousands of kernel data structures to check
that any paths that end at a function pointer will not invoke malicious
functionality. \osck uses internal kernel information to map memory
locations to data types, allowing it to check nodes in the graph of data
structures in isolation, and in any order (such as by their order in
memory), without maintaining the state necessary for a depth- or
breadth-first traversal.  Importantly, \osck does not trust the information
that it extracts from the kernel --- it will detect malfeasance for both a
malicious function pointer as well as for corruption of the type
information from the potentially malicious kernel.

\inktag extends this concept with the introduction of
\emph{paraverification}. Like paravirtualization, paraverification has an
untrusted operating system provide information essential for the \inktag
hypervisor to verify that the kernel is operating correctly in its
interactions with a trusted application. To do so, the hypervisor must
match OS modifications of application state (such as modifying page tables) to
application intent (such as a prior call to \codeid{mmap}). Previously, this
required the application and hypervisor to synchronize on a copy of the
application's memory map, and index and look up entries in response to
latency-critical events, such as page faults. However, the OS already
contains a copy of the memory map, in which it must also look up entries in
response to faults. With paraverification, the application need only record
entries in the memory map in an unordered array. The operating system is
then responsible for fast indexing and lookup during page faults, and
passing that information to the application and the hypervisor so that they
may validate updates. Again, key to this new set of responsibilities for
the untrusted OS is that although the application and hypervisor use
OS-provided information for efficiency and implementation simplicity, they
do not use it for correctness. \inktag can detect incorrect
paraverification information as well as it can detect other malicious
behavior by the operating system.

{\bigskip \noindent \bf Ongoing research}

\noindent Attributitis

\end{document}
