\section{Experience}
\href{https://katanagraph.ai}{\textbf{Katana Graph}}
\begin{outerlist}
\item[] \textit{Software Engineer}%
   \hfill \textbf{June to November 2021}
   \begin{outerlist}
      \item Initial design and implementation of indexing for Katana, a high
      performance graph database.
   \end{outerlist}
\end{outerlist}

\bigskip

\href{https://cloud.google.com/compute}{\textbf{Google Compute Engine}}
\begin{outerlist}
\item[] \textit{Senior Software Engineer}%
   \hfill \textbf{2017 to June 2021}
\item[] \textit{Software Engineer}%
   \hfill \textbf{September 2013 to 2017}
   \begin{outerlist}
      \item VM memory management and live migration
      \begin{outerlist}
         \item Ran early evaluations of TLB page size and NUMA scheduling
         effects on large memory VM workloads. Project lead for design,
         implementation and rollout of GCE VMs backed by hugetlbfs.
         \item Core contributor to GCE's live migration technology. Helped take
         post-copy memory migration from prototype to production. Authored
         numerous performance and reliability improvements in response to
         internal benchmarking and external customer pain. Rewrote GCE's VMM
         memory migration control for simplicity and testability.
      \end{outerlist}
      \item VM timekeeping
      \begin{outerlist}
         \item Responsible for identifying and addressing all issues
         related to timekeeping in GCE guests. Implemented consistency and
         policy for timekeeping across VM live migration.
         \item Led a project to provide Google's TrueTime service to GCE VMs.
         Designed, implemented and deployed a paravirtual interface allowing
         GCE guests to generate TrueTime intervals at native performance.
      \end{outerlist}
      \item Kernel/KVM development, qualification and rollout
      \begin{outerlist}
         \item Engineering lead for GCE production kernel qualification and
         rollout. Worked with teams across GCE and Google to identify and
         resolve long-tail bugs in the kernel (and occasionally CPU)
         deployed to production GCE hosts. GCE lead for upgrade from
         Linux upstream version 3.11 to 4.3.
         \item Contributed various gruntwork to GCE's response to
         Spectre/Meltdown, such as backporting mitigations.
         \item Contributor to landing many new platforms in GCE, especially new
         Intel microarchitectures. Lead for integrating new Cascadelake
	 CPUs. Designed and implemented stack for passing through host
	 non-volatile memory (Intel Apache Pass NVDIMMs) to GCE guests as a
	 customer-selectable addon.
      \end{outerlist}
      \item Core VMM development
      \begin{outerlist}
         \item Contributor to several areas across the core of GCE's C++
         hypervisor.
         \item Lead contributor for project to revamp the GCE hypervisor
         for new host platform topologies.
      \end{outerlist}
   \end{outerlist}
\end{outerlist}

\pagebreak

\href{http://www.cs.utexas.edu}{\textbf{The University of Texas at Austin,
Department of Computer Science}}
\begin{outerlist}

\item[] \textit{Research Assistant}%
    \hfill \textbf{May 2007 to August 2013}
    \begin{innerlist}
        \item TxLinux 2.6 
        \begin{innerlist}
           \item Worked on team to develop TxLinux, the first operating system
to use hardware transactional memory for synchronization
           \item Designed and implemented \emph{cooperative transactional
spinlocks}, forming the basis for large-scale automatic conversion of the
Linux kernel to use hardware transactions
           \item Published in ISCA '07 and SOSP '07
        \end{innerlist}
        \item TxLinux 2.4
        \begin{innerlist}
           \item Applied techniques from TxLinux 2.6 to Linux 2.4 to
demonstrate the benefit of hardware transactional memory for
improving coarse-grained synchronization performance in the operating
system
           \item Introduced \emph{transaction ordering}, a novel technique for
unifying hardware and software transactional memory in user applications
           \item Published in ASPLOS '09
        \end{innerlist}
        \item TxOS
        \begin{innerlist}
           \item Worked on team to develop TxOS, a version of Linux
providing \emph{system transactions} to ensure atomic, isolated and
consistent updates to diverse system resources
           \item Modified the ext3 file system to provide durable
transactional updates
           \item Published in SOSP '09
        \end{innerlist}
        \item Unvanish
        \begin{innerlist}
           \item Identified vulnerabilities in the Vanish system for
self-destructing data
           \item Designed and implemented Unvanish, a system to recover
data protected by Vanish
           \item Published in NDSS '10
         \end{innerlist}
         \item OSck
         \begin{innerlist}
            \item Designed and implemented a system for efficient
hypervisor-based kernel rootkit detection based on the Linux KVM hypervisor
            \item Published in ASPLOS '11
         \end{innerlist}
         \item InkTag
         \begin{innerlist}
            \item Designed and implemented InkTag, a hypervisor-based system
for securely executing applications under an untrusted operating system
            \item Published in ASPLOS '13, ASPLOS '16
         \end{innerlist}
    \end{innerlist}
\item[] \textit{Assistant Instructor}%
        \hfill \textbf{January 2012 to May 2012}
        \begin{innerlist}
           \item CS429: Computer Organization and Architecture
           \begin{innerlist}
              \item Served as instructor for an introductory undergraduate
computer architecture course (77 students)
              \item Developed new lectures and exams
           \end{innerlist}
        \end{innerlist}
\item[] \textit{Teaching Assistant}%
        \hfill \textbf{August 2007 to December 2007}
        \begin{innerlist}
           \item CS372H: Operating Systems Honors
           \begin{innerlist}
              \item Held office hours, graded undergraduate lab assignments and exams
              \item Developed new lab assignments on synchronization
              \item Designed a user study based on student experience and
performance with synchronization labs
              \item Study results later published in PPoPP '10
           \end{innerlist}
        \end{innerlist}
\end{outerlist}

\pagebreak

\href{https://research.microsoft.com}{\textbf{Microsoft Research}}
\begin{outerlist}

\item[] \textit{Intern}
    \hfill \textbf{June to August 2010}
    \begin{innerlist}
        \item Flat Datacenter Storage
        \begin{innerlist}
           \item Worked on team developing initial implementation of Flat
Datacenter Storage (FDS)
           \item Built initial implementation of failure recovery within
FDS
           \item FDS held records in both ``Indy'' and
``Daytona'' categories for the MinuteSort benchmark
(\href{https://www.sortbenchmark.org}{www.sortbenchmark.org}).
           \item Published in OSDI '12
        \end{innerlist}
     \end{innerlist}
\end{outerlist}
